\documentclass[a4paper]{article}

\setlength{\parskip}{2mm}
\newcommand{\tab}{~ \qquad}
\input{Algo1Macros}
\usepackage{caratula} % Version modificada para usar las macros de algo1 de ~> https://github.com/bcardiff/dc-tex

\begin{document}

\titulo{TP de Especificación}
\fecha{15 de Abril de 2022}
\materia{Algoritmos y Estructuras de Datos I}
\grupo{Grupo 2}

\newcommand{\dato}{\textit{Dato}}
\newcommand{\individuo}{\textit{Individuo}}


\integrante{De Galvagni, Constanza}{590/19}{constanzadegalvagni@gmail.com}
\integrante{Marin, Candela}{1405/21}{canmarin17@gmail.com}
\integrante{Pensotti, Santiago}{1579/21}{santipensotti@gmail.com}
\integrante{Levy, Luciano}{1473/21}{lucianolevy03@gmail.com}
\maketitle


\section{Definición de Tipos}
\begin{description}
	\item type Tiempo = \float \hspace{1cm}
	\item type Dist = \float
	\item type GPS = \float $\times$ \float
	\item type Recorrido = \TLista{GPS}
	\item type Viaje = $\TLista{Tiempo \times GPS}$
	\item type Nombre = \float $\times$ \float
	\item type Viaje = $\TLista{GPS \times GPS \times Nombre}$
\end{description}

\section{Constantes}
\begin{description}
\item aux dist(p1: GPS, p2: GPS) : Dist
\item \pred{esUnViajeValido}{\In v: Viaje}{
		(\forall i : \ent)(0 \leq i < \longitud v \wedge 0 \leq j < \longitud{v}-1  \implicaLuego gpsEnRango (v[j]_1,v[j+1]_1)) \wedge tiempoEnRango(v[i]_{0})
    }

\item \pred{gpsEnRango}{\In g1: GPS, \In g2: GPS}{
		( (-90.0 \leq g1_{0},g2_{0} \leq 90.0) \y (-180.0 \leq g1_{1},g2_{1} \leq 180.0) ) \y (  dist(g1,g2) \approx 20 )
    }
    
\item \pred{tiempoEnRango}{\In t: Viaje} {
    t_{0} > 0
    }

\item \pred{esUnRecorridoValido}{\In r: Recorrido}{
		(\forall j:\ent)(0 \leq j < \longitud{r}-1  \implicaLuego gpsEnRango (r[j],r[j+1]))
		}

\item DistanciaMaximaMetros = 444
    
\end{description}

\section{Problemas}
% el * le saca el numero de seccion
\subsection*{Ejercicio 1}

\begin{proc}{viajeValido}{\In v: Viaje, \Out res: $\bool$}{}
    \pre{\True}
	\post{esUnViajeValido(v)}
    
    
\end{proc}


\subsection*{Ejercicio 2}
\begin{proc}{recorridoValido}{\In v: Recorrido, \out result: $\bool$}{}
	\pre{esUnViajeValido(v)}
	\post{result = \True \leftrightarrow (\forall i:\ent) ( 0 \leq i < \longitud v \yLuego gpsEnRango(i) }

\end{proc}

\subsection*{Ejercicio 3}
\begin{proc}{enTerritorio}{\In v: Viaje, \In r: Dist, \Out res: $\bool$}{}
	\pre{esUnViajeValido(v) \wedge r>0}
	\post{res = True \leftrightarrow \dentroDeR(v,r)}

	\pred{dentroDeR}{v: Viaje,r: Dist}{((\forall i:\ent) (\forall j:\ent) ((0 \leq i , j < \longitud v \wedge j \neq i) \implicaLuego \frac{(dist(v[j]_1,v[i]_[1])}{1000} < 2r)}

\end{proc}

\subsection*{Ejercicio 4}
\begin{proc}{tiempoTotal}{\In v: Viaje,\Out t: Tiempo}{}
	\pre{esUnViajeValido(v)} 
	\post{t =  \sum_{i=0}^{\longitud v - 1} \IfThenElse {esMayor(v[i]_{0},v)}{v[i]_{0}}{(\IfThenElse {esMenor(v[i]_{0},v)}{v[i]_{0}}{0})}}
	

	\pred{esMayor}{t: Tiempo, v: Viaje, r: $\bool$}{ (\forall x:\ent)( 0 $\leq$ x $<$ \longitud v \implicaLuego t > v[x]_{0})
	}
	
	\pred{esMenor}{t: Tiempo, v: Viaje, r: $\bool$}{ (\forall x:\ent)( 0 $\leq$ x $<$ \longitud v \implicaLuego t < v[x]_{0})
		
	}
\end{proc}


\subsection*{Ejercicio 5}
\begin{proc}{distanciaTotal}{\In v: Viaje,\Out d: Dist}{}
	\pre{esUnViajeValido(v)} 
	\post{d =  \sum_{i=0}^{\longitud v - 1}distanciaSiguiente(v[i],v) }

	\aux{distanciaSiguiente}{in elem: (Tiempo \times GPS),v: Viaje}{\ent \\}
	{\sum_{i=0}^{\longitud v - 1}\IfThenElse {elem_0 = v[i]_0 - 20}{dist(elem_1,v[i]_1)}{0}}	

\end{proc}


\subsection*{Ejercicio 6}
\begin{proc}{excesoDeVelocidad(v)}{\In v: Viaje, \Out res: $\bool$}{}
	\pre{esUnViajeValido(v)}
	\post{res = \True \leftrightarrow  (\forall i : \ent)( 0 $\leq$ x $<$ \longitud v \implicaLuego distanciaSiguiente(v[i],v) < DistanciaMaximaMetros)}
	
\end{proc}

\subsection*{Ejercicio 7}
\begin{proc}{flota}{\In v: \TLista{Viaje},\In t_{0}: Tiempo, \In t_{f}: Tiempo,  \Out res: \ent}{}
	\pre{(\forall i : \ent)(0 \leq i < \longitud v \implicaLuego esUnViajeValido(v))}
	\post{res = {\sum_{i=0}^{|v|-1}}  \IfThenElse {(viajeEnElTiempoPedido (v[i], t_{0}, t_{f}))}{1}{0}}
	 
	\pred {ViajeEnElTiempoPedido} {v: Viaje, t_{0}: Tiempo, t_{f}: Tiempo}{ 
		(\exists i : \ent)(0 \leq i < \longitud v \yLuego t_{0} \leq v[i]_{0} \leq t_{f}) }
\end{proc}

\subsection*{Ejercicio 8}
\begin{proc}{recorridoNoCubierto}{\In v: Viaje, \In r: Recorrido,\In u: Dist,\Out res: \TLista{GPS}}{}
	\pre{esUnViajeValido(v) \wedge esUnRecorridoValido(r) \wedge u \geq 0} % HACER EL PREDICADO DE esUnRecorridoValido
	\post{(\forall i: \ent)(0 \leq i < \longitud {res} \implicaLuego \neg(cubierto(res[i],v,u)) \wedge 
	(\exists j: \ent)(0 \leq j < \longitud {res} \yLuego r[j] = res[i]))}

	\pred {cubierto} {e: GPS,v: Viaje,u: \ent}{
		(\exists i: \ent)(0 \leq i < \longitud v \yLuego dist(v[i]_1,e) \leq 1000 \times u)
		}
	
\end{proc}
\subsection*{Ejercicio 10}
\begin{proc}{regiones}{\In r: Recorrido,\In g: Grilla,\Out res: \TLista{NOMBRE}}{}
	\pre{recorridoValido(r)} % si el recorrido no es valido se puede llegar a indefinir
	\post{(\forall i : \ent)( 0 $\leq$ i $<$ \longitud v \implicaLuego res _i= (numeroFila(r_i,g),numeroColumna(r_i,g)}
	
	\aux{numeroFila}{in elem: (Tiempo \times GPS),g: Grilla}{\ent \\}
	{\sum_{i=0}^{\longitud g - 1}\IfThenElse {g[i]_0_0 <= elem_0_0<=g[i]_1_0 }{i}{0}}	

	\aux{numeroColumna}{in elem: (Tiempo \times GPS),g: Grilla}{\ent \\}
	{\sum_{i=0}^{\longitud g - 1}\IfThenElse {g[i]_0_1 <= elem_0_1<=g[i]_1_1 }{i}{0}}

\end{proc}


\subsection{Ejercicio 11}
\begin{proc}{cantidadDeSaltos}{\In g: Grilla,\In v: Viaje \Out res: \ent}{}
	\pre{esUnViajeValido(v) \y enGrilla(g,v)}
	\post{res = sumSaltos(g,v)}
	 
	 
	 \pred {enGrilla} {\In g: Grilla, \In v: Viaje} {
	(\forall i : \ent)(0 \leq i < |v|) \exists ((numeroFila(v[i],g) \y  (numeroColumna(v[i],g))
	}
	
	
	\pred {esUnSalto} {\In g: Grilla, \In v: Viaje,\In i: \ent}{ 
		(\exists j : \ent)(0 \leq j < |v|)  v[j]_{0} -  v[i]_{0} = 20 \yLuego ((numeroFila(v[j]_{1}) - numeroFila(v[i]_{1}) \geq 3) \vee (numeroColumna(v[j]_{1}) - numeroColumna(v[i]_{1}) \geq 3)) }	
		
	aux sumSaltos (in g: Grilla, in v: Viaje) {
	  \sum_{i=0}^{\longitud v - 1} \IfThenElse {esUnSalto(g,v,i)}{1}{0}
	}	
\end{proc}	

\end{document}
